% Format inherited from <MA1101R Cheatsheet 17/18 Sem 1 Finals>
% Original document is by Lee Yiyuan and Eugene Lim
% -------------------------------------------------------------
\usepackage{amssymb,amsmath,amsthm,amsfonts,bm,xcolor,enumitem,graphicx,overpic}
\usepackage{listings}
\usepackage{IEEEtrantools}
\usepackage{physics}
\usepackage{multicol,multirow}
\usepackage{calc}
\usepackage{ifthen}
\usepackage{ragged2e}
\usepackage[colorlinks=true,citecolor=blue,linkcolor=blue]{hyperref}

\geometry{top=.2in,left=.2in,right=.2in,bottom=.2in,a4paper}
\pagestyle{empty}
\makeatletter
\renewcommand{\section}{\@startsection{section}{1}{0mm}%
                                {-1ex plus -.5ex minus -.2ex}%
                                {0.5ex plus .2ex}%x
                                {\normalfont\large\bfseries}}
\renewcommand{\subsection}{\@startsection{subsection}{2}{0mm}%
                                {-1explus -.5ex minus -.2ex}%
                                {0.5ex plus .2ex}%
                                {\normalfont\normalsize\bfseries}}
\renewcommand{\subsubsection}{\@startsection{subsubsection}{3}{0mm}%
                                {-1ex plus -.5ex minus -.2ex}%
                                {1ex plus .2ex}%
                                {\normalfont\small\bfseries}}
\renewcommand*\env@matrix[1][*\c@MaxMatrixCols c]{%
	\hskip -\arraycolsep
	\let\@ifnextchar\new@ifnextchar
	\array{#1}}
\makeatother
\setcounter{secnumdepth}{0}
\setlength{\parindent}{0pt}
\setlength{\parskip}{0pt plus 0.5ex}

\newcommand{\matr}[1]{\bm{#1}}
\newcommand{\vect}[1]{\bm{#1}}
\newcommand{\adj}{\operatorname{\textbf{adj}}}
\newcommand{\lspan}{\operatorname{span}}
%\newcommand{\rank}{\operatorname{rank}}
\newcommand{\nullity}{\operatorname{nullity}}
\newcommand{\Ker}{\operatorname{Ker}}
%\newcommand{\norm}[1]{\left\lVert#1\right\rVert}

\DeclareMathOperator{\rref}{rref}

\theoremstyle{definition}
\newcommand{\thistheoremname}{}
\newtheorem*{genericthm*}{\thistheoremname}
\newenvironment{namedthm*}[1]
{\renewcommand{\thistheoremname}{#1}\begin{genericthm*}}
{\end{genericthm*}}