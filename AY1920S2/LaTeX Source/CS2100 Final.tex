% Author: Howard Liu
\documentclass[a4paper]{article}
\usepackage[
a4paper, left=2cm, right=2cm, top=2cm, bottom=2cm, portrait
]{geometry}
\usepackage{multicol}
\usepackage{ragged2e}
\usepackage{amsmath}

% STYLE DEFINITION %
\pagestyle{empty}
\makeatletter
\renewcommand{\section}{\@startsection{section}{1}{0mm}%
	{-1ex plus -.5ex minus -.2ex}%
	{0.5ex plus .2ex}%x
	{\normalfont\large\bfseries}}
\renewcommand{\subsection}{\@startsection{subsection}{2}{0mm}%
	{-1explus -.5ex minus -.2ex}%
	{0.5ex plus .2ex}%
	{\normalfont\normalsize\bfseries}}

\begin{document}
	\pagenumbering{gobble}              % No page numbers
	% TITLE %
	\begin{center}
		{\large CS2100 Computer Organisation Cheatsheet}\\{for Final of AY 19/20 Semester 2, by Howard Liu}
	\end{center}

    \begin{justifying}
    	\setlength\parindent{0em}           % No indents at start of paragraphs
    	\setlength\parskip{0.5em}           % Increase spacing between paragraphs
    	\begin{multicols}{2}
    		\section{IEEE 754 Floating Point}
    		
    		Given a 32-bit single precision IEEE 754 floating point number \textit{A}, we have the three parts:
    		
    		\begin{itemize}
    			\item A[31:31] - Sign bit, 0 for +ve, 1 for -ve;
    			\item A[30:23] - Exponent bits (8 bits)
    			\item A[22:0]  - Mantissa bits (24 bits)
    		\end{itemize}
    	
    	    From a decimal form, we can calculate the IEEE 754 representation in the following steps (taking number \textbf{85.125} as an example):
    	    
    	    \begin{enumerate}
    	    	\item Covert integer part and decimal part into binary
    	    	    \begin{enumerate}
    	    	      	\item $85 = (1010101)_2$
    	    	      	\item $.125 = (.001)_2$
    	    	    \end{enumerate}
        	    \item We have $1010101.001$ now. Make it only have one bit in the integer part, hence we do right shift for 6 bits. ($1010101.001 = 1.010101001 * 2^6$)
        	    \item We calculate the exponent bit. For single precision, $\text{exponent} = 127 + \text{power of 2}$ . Hence we have 133 for this example. ($133 = (10000101)_2$)
        	    \item Put everything together with 0 fills at the back, we have the binary representation in IEEE 754 format:\\
        	        (0 1000101 010101001 00000000000000)$_2$ = 0x42AA4000
        	    \item Sometimes numbers cannot be represented precisely in binary (e.g. \textbf{0.3}), we do the binary conversion until we have the full 23-bit Mantissa.
    	    \end{enumerate}
        
        
            \section{Binary convertion}
            
            To convert a decimal point to binary, taking \textbf{0.125} as an example, we do this:
            
            \begin{center}
            	\begin{tabular}{lll}
            		\hline
            		Decimal Number & Result & Binary \\
            		\hline
            		$0.125 * 2$ & $\textbf{0}.25$ & $0$ \\
            		$0.25 * 2$ & $\textbf{0}.50$ & $0$ \\
            		$0.50 * 2$ & $\textbf{1}.00$ & $1$ \\
            		$0.00 * 2$ & $\textbf{0}.00$ & $0$ \\
            		\hline
            	\end{tabular}
            \end{center}
       
            * Binary number is read from top to bottom. In this case, $.125 = (.001)_2$ .
            
            
            \section{Negating Numbers}
            
            To negate a binary number, we have
            \begin{itemize}
                \item Sign-and-Magnitude: just flip MSB to negate a number ($0001_2 \rightarrow 1001_2$)
                \item 1s Complement: flip every bits to negate a number ($0001_2 \rightarrow 1110_2$)
                \item 2s Complement: add 1 to \textbf{LSB} of 1s Complement ($0001.01_2 \rightarrow 1110.10_2 \rightarrow 1110.11_2$)
            \end{itemize}
        
            We can detect overflow if either happens:
            \begin{itemize}
                \item +ve add +ve become -ve
                \item -ve add -ve become +ve
            \end{itemize}
        
            When performing binary addition on \textbf{1s-complement} numbers, if there is a carry out of the MSB, add 1 to the result and dispose the MSB. We do \textbf{NOT} need to do so for addition on \textbf{2s-complement} numbers.
                        
            
            \section{Boolean algebra laws and theorems}
            
            Duality: if the AND/OR operators and identity elements 0/1 in a Boolean equation are interchanged, it remains valid (but brackets are to be added when necessary). (e.g. $X + X \cdot Y = X \rightarrow X \cdot (X + Y) = X$)
            
            Laws:
            \begin{itemize}
                \item Identity: $A + 0 = 0 + A = A$
                \item Inverse/Complement: $A + A' = 1$
                \item Commutative: $A + B = B + A$
                \item Associative: $A + (B + C) = (A + B) + C$
                \item Distributive: $A + (B \cdot C) = A \cdot B + A \cdot C$
            \end{itemize}
        
            Theorems:
            \begin{itemize}
                \item Idempotency: $X + X = X$
                \item One element/Zero element: $X + 1 = 1$
                \item Involution: $(X')' = X$
                \item Absorption 1: $X + X \cdot Y = X$
                \item Absorption 2: $X + X' \cdot Y = X + Y$
                \item DeMorgans': $(X + Y)' = X' \cdot Y'$
                \item Consensus: $X \cdot Y + X' \cdot Z + Y \cdot Z = X \cdot Y + X' \cdot Z$
            \end{itemize}
          
          
            \section{Flip-flop reference}
            
            Characteristic Tables:
            
            \begin{multicols}{2}
                \begin{center}
                    \begin{tabular}{cc|c}
                        J & K & Q$^+$ \\
                        \hline
                        0 & 0 & Q \\
                        0 & 1 & 0 \\
                        1 & 0 & 1 \\
                        1 & 1 & Q(t)' \\
                    \end{tabular}
                    
                    \begin{tabular}{c|c}
                        D & Q$^+$ \\
                        \hline
                        0 & 0 \\
                        1 & 1 \\
                    \end{tabular}
                
                    \begin{tabular}{cc|c}
                        S & R & Q$^+$ \\
                        \hline
                        0 & 0 & Q \\
                        0 & 1 & 0 \\
                        1 & 0 & 1 \\
                        1 & 1 & Unpredictable \\
                    \end{tabular}
                
                    \begin{tabular}{c|c}
                        T & Q$^+$ \\
                        \hline
                        0 & Q \\
                        1 & Q' \\
                    \end{tabular}
                \end{center}
            \end{multicols}
           
            Excitation Tables:
            
            \begin{multicols}{2}
                \begin{center}
                    \begin{tabular}{cc|cc}
                        Q & Q$^+$ & J & K \\
                        \hline
                        0 & 0 & 0 & X \\
                        0 & 1 & 1 & X \\
                        1 & 0 & X & 1 \\
                        1 & 1 & X & 0 \\
                    \end{tabular}
                
                    \begin{tabular}{cc|c}
                        Q & Q$^+$ & D \\
                        \hline
                        0 & 0 & 0 \\
                        0 & 1 & 1 \\
                        1 & 0 & 0 \\
                        1 & 1 & 1 \\
                    \end{tabular}
                
                    D = Q$^+$
                
                    \begin{tabular}{cc|cc}
                        Q & Q$^+$ & S & R \\
                        \hline
                        0 & 0 & 0 & X \\
                        0 & 1 & 1 & 0 \\
                        1 & 0 & 0 & 1 \\
                        1 & 1 & X & 0 \\
                    \end{tabular}
                
                    \begin{tabular}{cc|c}
                        Q & Q$^+$ & T \\
                        \hline
                        0 & 0 & 0 \\
                        0 & 1 & 1 \\
                        1 & 0 & 1 \\
                        1 & 1 & 0 \\
                    \end{tabular}
                
                    T = Q $\oplus$ Q$^+$
                \end{center}
            \end{multicols}
            
            
            \section{Notes about pipeline}
            Considering instruction A and B. B is just the instruction executed after A.
            \begin{itemize}
                \item A pipeline stage of B will always behind the same stage for A, even if the resource is not occupied and there is no data dependency / control hazard. (e.g. D stage of instruction B cannot be executed at cycle 3, even D is vacant)
                \begin{center}
                    \begin{tabular}{c|cccccccc}
                        Inst & 1 & 2 & 3 & 4 & 5 & 6 & 7 & 8 \\
                        \hline
                        A    & F &   &   & D & E & M & W &   \\
                        B    &   & F &   &   & D & E & M & W \\
                    \end{tabular}
                \end{center}
                \item For $N$ instructions, ideal 5-stage pipeline takes $N+2$ cycles to execute.
                \item Delays introduced when no forwarding, no early branching and no branch prediction.
                \begin{itemize}
                    \item Data dependency: \textbf{2 cycles}
                    \item Control hazard (beq/bne, j): \textbf{3 cycles}
                \end{itemize}
                \item Delays introduced when there is forwarding: \textbf{1 cycle} for data dependency after \textbf{lw}.
                \item Data dependency for \textbf{beq/bne} instruction can lead to delay \textbf{when forwarding and early branching is implemented}. This is because data is required in stage 2 (D), not stage 3 (E), of the branching instruction.
                \item When branch prediction is wrong, flush everything including the correct ones to be executed. This is especially true when there is \textbf{no early branching}. Considering the following example: assuming not taken is used, but actually A branched to C. C and D is executed once again after everything flushed.
                \begin{center}
                    \begin{tabular}{l|cccccccc}
                        Inst       & 1 & 2 & 3 & 4 & 5 & 6 & 7 & 8 \\
                        \hline
                        A (beq)    & F & D & E & M & W &   &   &   \\
                        B          &   & F & D & E & X & X &   &   \\
                        C          &   &   & F & D & X & X & X &   \\
                        D          &   &   &   & F & X & X & X & X \\
                        C (again)  &   &   &   &   & F & D & E & M \\
                    \end{tabular}
                \end{center}
            \end{itemize}
        
            
            \section{Cache}
            
            Average Access Time formula:
            $$ \text{Hit rate} \times \text{Hit time} + (1 - \text{Hit rate}) \times \text{Miss penality}$$
            
            
            \section{Quick reference}
            
            if (a $<$ b) \{...\} else \{...\}
            
            \begin{center}
                \begin{tabular}{|l|}
                    \hline
                    slt \$t0, \$a, \$b \\
                    beq \$t0, \$zero, else \\
                    \hline
                \end{tabular}
            \end{center}
            
            if (a $>=$ b) \{...\} else \{...\} : flip a and b in the above example.
            
            
            \section{Miscellaneous notes}
            
            \begin{itemize}
                \item Zero-extension and Sign-extension is different especially in the case that the number starts with 1 in binary.
                \item MIPS does not have \textbf{not} instruction. We use \textbf{nor \$1, \$1, \$zero} to do bitwise not.
            \end{itemize}
    	\end{multicols}
    \end{justifying}
\end{document}