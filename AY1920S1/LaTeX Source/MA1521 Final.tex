\documentclass[10pt,landscape]{article}
\usepackage[landscape]{geometry}
% Format inherited from <MA1101R Cheatsheet 17/18 Sem 1 Finals>
% Original document is by Lee Yiyuan and Eugene Lim
% -------------------------------------------------------------
\usepackage{amssymb,amsmath,amsthm,amsfonts,bm,xcolor,enumitem,graphicx,overpic}
\usepackage{listings}
\usepackage{IEEEtrantools}
\usepackage{physics}
\usepackage{multicol,multirow}
\usepackage{calc}
\usepackage{ifthen}
\usepackage[colorlinks=true,citecolor=blue,linkcolor=blue]{hyperref}
\usepackage{ragged2e}

\geometry{top=.2in,left=.2in,right=.2in,bottom=.2in,a4paper}
\pagestyle{empty}
\makeatletter
\renewcommand{\section}{\@startsection{section}{1}{0mm}%
                                {-1ex plus -.5ex minus -.2ex}%
                                {0.5ex plus .2ex}%x
                                {\normalfont\large\bfseries}}
\renewcommand{\subsection}{\@startsection{subsection}{2}{0mm}%
                                {-1explus -.5ex minus -.2ex}%
                                {0.5ex plus .2ex}%
                                {\normalfont\normalsize\bfseries}}
\renewcommand{\subsubsection}{\@startsection{subsubsection}{3}{0mm}%
                                {-1ex plus -.5ex minus -.2ex}%
                                {1ex plus .2ex}%
                                {\normalfont\small\bfseries}}
\renewcommand*\env@matrix[1][*\c@MaxMatrixCols c]{%
	\hskip -\arraycolsep
	\let\@ifnextchar\new@ifnextchar
	\array{#1}}
\makeatother
\setcounter{secnumdepth}{0}
\setlength{\parindent}{0pt}
\setlength{\parskip}{0pt plus 0.5ex}

\newcommand{\matr}[1]{\bm{#1}}
\newcommand{\vect}[1]{\bm{#1}}
\newcommand{\adj}{\operatorname{\textbf{adj}}}
\newcommand{\lspan}{\operatorname{span}}
%\newcommand{\rank}{\operatorname{rank}}
\newcommand{\nullity}{\operatorname{nullity}}
\newcommand{\Ker}{\operatorname{Ker}}
%\newcommand{\norm}[1]{\left\lVert#1\right\rVert}

\DeclareMathOperator{\rref}{rref}

\theoremstyle{definition}
\newcommand{\thistheoremname}{}
\newtheorem*{genericthm*}{\thistheoremname}
\newenvironment{namedthm*}[1]
{\renewcommand{\thistheoremname}{#1}\begin{genericthm*}}
{\end{genericthm*}}

% Format inherited from <MA1101R Cheatsheet 17/18 Sem 1 Finals>
% Original document is by Tysng
% Original link: https://github.com/tysng/ma1521-cheatsheet
% PDE Removed given it is not examinable for 1920S1

% Turn off header and footer
\pagestyle{empty}

% Redefine section commands to use less space
\makeatletter
\renewcommand{\section}{\@startsection{section}{1}{0mm}%
                                {-1ex plus -.5ex minus -.2ex}%
                                {0.5ex plus .2ex}%x
                                {\normalfont\large\bfseries}}
\renewcommand{\subsection}{\@startsection{subsection}{2}{0mm}%
                                {-1explus -.5ex minus -.2ex}%
                                {0.5ex plus .2ex}%
                                {\normalfont\normalsize\bfseries}}
\renewcommand{\subsubsection}{\@startsection{subsubsection}{3}{0mm}%
                                {-1ex plus -.5ex minus -.2ex}%
                                {1ex plus .2ex}%
                                {\normalfont\small\bfseries}}
\makeatother

% Define BibTeX command
\def\BibTeX{{\rm B\kern-.05em{\sc i\kern-.025em b}\kern-.08em
    T\kern-.1667em\lower.7ex\hbox{E}\kern-.125emX}}

% print only section numbers
\setcounter{secnumdepth}{1}


\setlength{\parindent}{0pt}
\setlength{\parskip}{0pt plus 0.5ex}

%My Environments
\newtheorem{example}[section]{Example}


% -----------------------------------------------------------------------

\begin{document}
\raggedright
\footnotesize
\begin{multicols}{3}


% multicol parameters
% These lengths are set only within the two main columns
%\setlength{\columnseprule}{0.25pt}
\setlength{\premulticols}{1pt}
\setlength{\postmulticols}{1pt}
\setlength{\multicolsep}{1pt}
\setlength{\columnsep}{2pt}

\begin{flushleft}
\large{
    \underline{MA1521 Cheat Sheet} \\
    \texttt{by Howard Liu based on versions of Tysng} \\
    AY2019/20 Semester 1}
\end{flushleft}

% ------------------------------ACTUAL CONTENT-----------------------------------

\section{MF26 Magic}

\subsection{Trigo}
\begin{gather*}
   \sin (A \pm B) = \sin A \cos B \pm \cos A \sin B \\
   \cos (A \pm B) = \cos A \cos B \mp \sin A \sin B \\
   \tan (A \pm B) = \frac{\tan A \pm \tan B}{1 \mp \tan A \tan B} \\
   \sin 2A = 2 \sin A \cos A \\
   \cos 2A = \cos ^ 2 A - \sin ^ 2 A = 2\cos^2A - 1 = 1 - 2\sin^2A \\
   \tan 2A = \frac{2 \tan A}{1- \tan^2 A} \\
   \sin P + \sin Q = 2 \sin \frac{1}{2} (P + Q) \cos \frac{1}{2} (P - Q) \\
   \sin P - \sin Q = 2 \cos \frac{1}{2} (P + Q) \sin \frac{1}{2} (P - Q) \\
   \cos P + \cos Q = 2 \cos \frac{1}{2} (P + Q) \cos \frac{1}{2} (P - Q) \\
   \cos P - \cos Q = -2 \sin \frac{1}{2} (P + Q) \sin \frac{1}{2} (P - Q) \\
   \frac{\sin A}{a} = \frac{\sin B}{b} = \frac{\sin C}{c} \\
   a^2 =  b^2 + c^2 - 2bc \cos A
\end{gather*}

\subsection{Derivatives}
\begin{gather*}
   \frac{d}{dx} \sin ^{-1} x = \frac{1}{\sqrt{1-x^2}} \\
   \frac{d}{dx} \cos ^{-1} x = -\frac{1}{\sqrt{1-x^2}} \\
   \frac{d}{dx} \tan ^{-1} x = \frac{1}{1+x^2} \\
   \frac{d}{dx} \csc x = - \csc x \cot x \\
   \frac{d}{dx} \sec x = \sec x \tan x \\
   \frac{d}{dx} \tan x = \sec^2 x \\
   \frac{d}{dx} \cot x = -\csc^2 x
\end{gather*}

\subsection{Integrals}
Take note of the absolute sign, and always remember to $+c$
\begin{gather*}
   \int \frac{1}{x^2 + a^2} dx = \frac{1}{a} \tan ^{-1} (\frac{x}{a}) + c \\ 
   \int \frac{1}{\sqrt{a^2 - x^2}} dx = \sin ^{-1} (\frac{x}{a}) + c \\ 
   \int \frac{1}{x} dx = \ln \abs{x} + c \\
   \int \frac{x}{\sqrt{a^2 - x^2}} dx = -\sqrt{a^2 - x^2} + c \\
   \int x^n e^x dx = x^n e^x - n x^{n-1} e^x + n (n-1) x^{n-2}e^x - \dots \pm n! e^x + c
\end{gather*}


\section{Basics}
\subsection{Extreme Values}
Points where $f$ can have an extreme value:
\begin{itemize}
   \item Interior point where $f'(x) = 0$
   \item Interior points where $f'(x)$ doesn't exist
   \item End points of the domain of $f$
\end{itemize}

\subsection{L'Hospital's Rule}
The $\frac{0}{0}$ form: (1) $f$ and $g$ are differentiable in a neighborhood of $x_0$, 
(2) $f(x_0) = g(x_0) = 0$, (3) $g'(x) \neq 0$ except possibly at $x_0$
\begin{gather*}
   \lim_{x\to x_0} \frac{f(x)}{g(x)} = \lim_{x\to x_0} \frac{f'(x)}{g'(x)}
\end{gather*}
E.g. $\lim_{x\to 0} \frac{3x - \sin x }{x} = \frac{3-\cos x }{1} \rvert_{x = 0} = 2$

The $\frac{\infty}{\infty}$ form: when $x \to a$, $f(x), g(x) \to \infty$,
and both differentiable,
\begin{gather*}
   \lim_{x\to a} \frac{f(x)}{g(x)} = \lim_{x\to a} \frac{f'(x)}{g'(x)}
\end{gather*}

Else, change to these two forms. (e.g $\lim_{x\to 0^+} x \cot x = \lim_{x\to 0^+} \frac{x}{\tan x} 
= \lim_{x\to 0^+} \frac{1}{\sec^2 x} = 1$ )

\subsection{Fundamental Theorem of Calculus}
\begin{gather*}
   \frac{d}{d\Box} \int_{c}^{\Box} f(t) dt = f(\Box)
\end{gather*}


\section{Series}
\subsection{Geometric Series}
Sum: $S_n = a \frac{1-r^n}{1-r}$, $r\neq 1$

Ratio test: $\lim_{n\to\infty} = \abs{\frac{a_{n+1}}{a_n}} = \rho$; 
\begin{enumerate}
   \item $\rho < 1$: converge; 
   \item $\rho > 1$: diverge;
   \item $\rho = 1$, no conclusion;
\end{enumerate}
For convergent series: $S_n \to \frac{a}{1-r}$

\subsection{Power Series}
$\sum_{n = 0}^{\infty} c_n(x-a)^n = c_0 + c_1(x-a) + c_2(x-a)^2 + \ldots + c_n(x-a)^n + \ldots$, where $a$ is the center of the power series

Convergence: $n\to\infty, S_n \to k$
\begin{enumerate}
   \item $\sum c_n(x-a)^n$ converges at $x=a$ and diverges elsewhere
   \item $h \in \mathbb{Z}$ that the series only converges in $(a-h, a + h)$
   \item converges for every $x$
\end{enumerate}

\subsection{Finding Radius of Convergence}
Apply ratio test and find
\begin{gather*}
   M = \lim_{n\to\infty} \abs{\frac{u_{n+1}}{u_n}}
   M < 1
\end{gather*}
and transform it to the form of $\abs{x-a} < b$; $a$ is the center, $b$ is the RoC

Or, if the series converges for all $x$, the RoC is $\infty$; if it only converges at $a$, the RoC is 0;

Some magic:
\begin{gather*}
   \frac{1}{1-\Box} = \sum_{n = 0 }^{\infty} \Box^n, \abs{\Box} < 1
\end{gather*}


\subsection{Taylor Series}
of $f$ at $a$:
\begin{gather*}
   \sum_{k=0}^{\infty} \frac{f^{(k)}(a)}{k!} (x-a)^k = f(a) + f'(a)(x-a) + \ldots \\
   + \frac{f^{(n)}(a)}{n!} (x-a)^n +\ldots \\
   e^x = \sum_{n=0}^{\infty} \frac{x^n}{n!} \\
   \sin x = \sum_{n=0}^{\infty} \frac{(-1)^n x ^{2n+1}}{(2n+1)!} \\
   \cos x = \sum_{n=0}^{\infty} \frac{(-1)^n x ^{2n}}{(2n)!} \\
   \ln (1+x )= \sum_{n=1}^{\infty} \frac{(-1)^{n-1}x^n}{n} \\
   \tan^{-1} x = \sum_{n=0}^{\infty} \frac{(-1)^n x^{2n+1}}{2n+1}
\end{gather*}

\subsection{Finding a specific high order derivative}
\begin{enumerate}
   \item given $\int f dx$
   \item evaluate f in polynomial form and integrate the polynomial form
   \item Compare the coefficient with the item that contains $f^{(100)}(0)$ in the Taylor expansion
\end{enumerate}

\subsection{Rules of Series}
\begin{gather*}
   \int \sum_{n=0}^{\infty} \Box dx = \sum_{n=0}^{\infty} \int \Box dx \\
   \frac{d}{dx} \sum_{n=0}^{\infty} \Box = \sum_{n=0}^{\infty} \frac{d}{dx} \Box \\
   x^k e^x = x^k \sum_{n=0}^{\infty} \frac{x^n}{n!} = \sum_{n=0}^{\infty} \frac{x^{n+k}}{n!} \text{ (Note: $x^k$ is a constant in the series)}
\end{gather*}


\section{Vectors}
Angle between two vectors: $\cos \theta = \frac{x_1 x_2 + y_1 y_2 + z_1 z_2}{\norm{\vec{v_1}} \norm{\vec{v_2}}}$ \\

Perpendicular vectors: $\vec{v_1} \cdot \vec{v_2} = 0$


\section{Partial Differentiation}
\begin{gather*}
f_{xy} (a,b) = f_{yx} (a,b) \\
\frac{dz}{dt} = \frac{\partial f}{\partial x} \frac{dx}{dt} + \frac{\partial f}{\partial y} \frac{dy}{dt} \\
\end{gather*}

\subsection{Directional Derivative}
For \textbf{unit vector} $u = u_1 \vec{i} + u_2 \vec{j}$, we have:

$$D_{\vec{u}}f(a,b) = f_x(a, b) \cdot u_1 + f_y(a,b) \cdot u_2$$

Gradient Vector: 

$$\nabla f = f_x \vec{i} + f_y \vec{j}$$

Relation between $D_{\vec{u}}f(a,b)$ and $\nabla f$:

$$D_{\vec{u}}f(a,b) = \nabla f(a,b) \cdot \vec{u} = \norm{\nabla f(a,b)} \cos \theta$$

Some characteristics:
\begin{itemize}
	\item $f$ increases most rapidly in $\nabla f(a,b)$ and decreases most rapidly in $ -\nabla f(a,b)$.
	\item Max value of $D_{\vec{u}}f(a,b) = \norm{\nabla f(a,b)}$ when $\vec{u}$ and $\nabla f$ in the same direction, since $\cos \theta = 0$
	\item Increment in $f$ (approx.): $\Delta f \approx [D_{\vec{u}} f(\vec{p})] (\Delta t)$, where $p$ is the point to measure the increment and $u$ is the unit vector of direction.
\end{itemize}

\subsection{Finding $D_{\vec{u}}f$}
\begin{enumerate}
   \item Find the direction $\vec{p}$
   \item Find the unit vector $\vec{u} = \frac{\vec{p}}{\norm{\vec{p}}}$
   \item Find $\nabla f$, then find $D_{\vec{u}} f = \nabla f \cdot \vec{u}$
\end{enumerate}

\subsection{Critical Points}
A point of $f$ that satisfies either is a critical point:
\begin{enumerate}
   \item $f_x (a,b) = 0$ and $f_y(a,b) = 0$
   \item $f_x (a,b)$ or $f_y(a,b)$ doesn't exist
\end{enumerate}

Perform Second Derivative Test: let $f_x(a,b) = 0$ and $f_y(a,b) = 0$
$$ D = f_{xx}(a,b) f_{yy} (a,b) - f_{xy} (a,b)^2 $$

\begin{itemize}
   \item $D > 0, f_{xx} >0$, f has a local minimum at (a,b)
   \item $D > 0, f_{xx} <0$, f has a local maximum at (a,b)
   \item $D < 0$, f has a saddle point at (a,b)
   \item $D = 0$, no conclusion
\end{itemize}


\section{Double Integrals}
For a region $R$ s.t. $a \leq x \leq b$ and $c \leq y \leq d$, volume is given by:
\begin{gather*}
\int\!\int_R\! f(x, y)\, \mathrm{d}A = \int_c^d\!\int_a^b\! f(x, y)\, \mathrm{d}x \mathrm{d}y = \int_a^b\!\int_c^d\! f(x, y)\, \mathrm{d}y \mathrm{d}x
\end{gather*}
if $f(x, y) = g(x)h(y)$,
then $$\int\!\int_R f(x,y) \, \mathrm{d}A = (\int_a^b\!g(x)\, \mathrm{d}x) (\int_c^d\!h(y)\, \mathrm{d}y)$$

\subsection{Rectangular Regions}
Express horizontal/vertical bounds as a function $g(x)$ or $h(y)$ \\
Type A(top and bottom are curves)
$$
\int_a^b\!
\left [
\int_{g_1(x)}^{g_2(x)}\!f(x, y)\, \mathrm{d}y
\right ] \mathrm{d}x $$
Type B(left and right are curves) $$\int_c^d\!
\left [
\int_{h_1(y)}^{h_2(y)}\!f(x, y)\, \mathrm{d}x
\right ] \mathrm{d}y$$

\subsection{Polar Coordinates}
$R$: $a \leq r \leq b$, $\alpha \leq \theta \leq \beta$
$$
\int\!\int_R\! f(x, y)\, \mathrm{d}A = 
\int_\alpha^\beta\!\int_a^b\!f(r\cos\theta, r\sin\theta)r\, \mathrm{d}r\mathrm{d}\theta
$$

\subsection{Surface Area}
$$S = \int\!\int_R\! \sqrt{(\frac{\partial z}{\partial x})^2 + (\frac{\partial z}{\partial y})^2 + 1 }\, \mathrm{d}A$$


\section{Ordinary Differential Equation}
\subsection{Separable Equations}
\begin{gather*}
   M(x) - N(y) y' = 0 \implies \int M(x) dx = \int N(y) dy + c
\end{gather*}

\subsection{Reduction to Separable Form}
Let $v = y / x \implies y = xv \rightarrow y' = v + xv'$, transform equations of $y' = g(\frac{y}{x})$ 
to $v + xv' = g(v) $ such that
\begin{gather*}
   \frac{dv}{g(v) - v} = \frac{dx}{x}
\end{gather*}
Similarly, $y' = f(ax + by + c)$ can be solved by $u = ax + by + c$

\subsection{Linear First Order ODE}
To solve $y' + Py = Q$: find integration factor 
$$R = e^{\int P dx}$$
Then, answer $$y = \frac{1}{R} \int RQ dx$$

\subsection{Reduction to Linear Form}
A Bernoulli equation: $y' + P(x)y = Q(x) y^n$, where $n \in \mathbb{R}$; \\ 
(When $n = -1$, try Reduction to Separable Form)
To solve it, let $v = y^{1-n}$; \\ 
Find and express $dv / dx$ in $dy / dx$; find $dy/dx$ and sub that in original equation; transform into 
$$v' + (1 - n)Pv = Q(1-n)$$
and solve the linear ODE.


\subsection{Homogeneous Linear Second Order DE}
For $y'' + a y' + by = 0$, the characteristic equation is $\lambda^2 + a \lambda + b = 0$

Find $\Delta = a ^2 -4b$:

\begin{enumerate}
   \item $\Delta > 0$, $y = c_1e^{\lambda_1x} + c_2e^{\lambda_2x}$
   \item $\Delta = 0$, $y = (c_1 + c_2 x) e^{-\frac{ax}{2}} $
   \item $\Delta < 0$, it has two complex roots;$ \lambda_1 = \alpha + \beta i, \lambda_2 = \alpha -\beta i$; $y = c_1 e^{\alpha x} \cos \beta x + c_2 e^{\alpha x} \sin \beta x$
\end{enumerate}
where,
\begin{gather*}
   \lambda_1 = \frac{1}{2} (-a+\sqrt{a^2-4b}) \\
   \lambda_2 = \frac{1}{2} (-a-\sqrt{a^2-4b})
\end{gather*}


\section{Modelling}
\subsection{Population Growth}
Malthus's Model: not an accurate representation
\begin{gather*}
   \frac{dN}{dt} = kN, k = B - D \\
   N(t) = N_0 e^{kt} 
\end{gather*}

\subsection{Logistic Model}
Let $D = sN$, where $s$ is a constant and $B$ is birth rate per capita:
$$ \frac{dN}{dt} = BN - DN = BN - sN^2 $$

The curve approaches carrying capacity $N = B/S$; point of inflection is at $N = B/2s$

$$ N = \frac{N_{\infty}}{1 + (\frac{N_{\infty}}{N_0} - 1) e^{-Bt}}, N_{\infty} = \frac{B}{s} $$

\subsection{Harvesting}
Let $E$ is fish caught per year, similar to the above model:
$$\frac{dN}{dt} = BN - sN^2 - E$$

Desirable result: $E < \frac{B^2}{4s}$, approaches the second root $\beta_2 = \frac{B + \sqrt{B^2 - 4Es}}{2s}$, when $\frac{dN}{dt} = 0$ 

\subsection{Strategies}
\begin{itemize}
   \item When given $dx/dt$, find $x$ that $dx/dt = 0$, draw out the axis, determine the sign of $dx/dt$ within
      each region, and find the flow (+ to the right, - to the left)
   \item To find E, draw the graph without E and find the line of symmetry; use the product of the roots to find E;
\end{itemize}

\end{multicols}
\end{document}

