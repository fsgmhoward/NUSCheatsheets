\documentclass[10pt,portrait]{article}
\usepackage[portrait]{geometry}
% Format inherited from <MA1101R Cheatsheet 17/18 Sem 1 Finals>
% Original document is by Lee Yiyuan and Eugene Lim
% -------------------------------------------------------------
\usepackage{amssymb,amsmath,amsthm,amsfonts,bm,xcolor,enumitem,graphicx,overpic}
\usepackage{listings}
\usepackage{IEEEtrantools}
\usepackage{physics}
\usepackage{multicol,multirow}
\usepackage{calc}
\usepackage{ifthen}
\usepackage[colorlinks=true,citecolor=blue,linkcolor=blue]{hyperref}
\usepackage{ragged2e}

\geometry{top=.2in,left=.2in,right=.2in,bottom=.2in,a4paper}
\pagestyle{empty}
\makeatletter
\renewcommand{\section}{\@startsection{section}{1}{0mm}%
                                {-1ex plus -.5ex minus -.2ex}%
                                {0.5ex plus .2ex}%x
                                {\normalfont\large\bfseries}}
\renewcommand{\subsection}{\@startsection{subsection}{2}{0mm}%
                                {-1explus -.5ex minus -.2ex}%
                                {0.5ex plus .2ex}%
                                {\normalfont\normalsize\bfseries}}
\renewcommand{\subsubsection}{\@startsection{subsubsection}{3}{0mm}%
                                {-1ex plus -.5ex minus -.2ex}%
                                {1ex plus .2ex}%
                                {\normalfont\small\bfseries}}
\renewcommand*\env@matrix[1][*\c@MaxMatrixCols c]{%
	\hskip -\arraycolsep
	\let\@ifnextchar\new@ifnextchar
	\array{#1}}
\makeatother
\setcounter{secnumdepth}{0}
\setlength{\parindent}{0pt}
\setlength{\parskip}{0pt plus 0.5ex}

\newcommand{\matr}[1]{\bm{#1}}
\newcommand{\vect}[1]{\bm{#1}}
\newcommand{\adj}{\operatorname{\textbf{adj}}}
\newcommand{\lspan}{\operatorname{span}}
%\newcommand{\rank}{\operatorname{rank}}
\newcommand{\nullity}{\operatorname{nullity}}
\newcommand{\Ker}{\operatorname{Ker}}
%\newcommand{\norm}[1]{\left\lVert#1\right\rVert}

\DeclareMathOperator{\rref}{rref}

\theoremstyle{definition}
\newcommand{\thistheoremname}{}
\newtheorem*{genericthm*}{\thistheoremname}
\newenvironment{namedthm*}[1]
{\renewcommand{\thistheoremname}{#1}\begin{genericthm*}}
{\end{genericthm*}}

% Format inherited from <MA1101R Cheatsheet 17/18 Sem 1 Finals>
% Original document is by Lee Yiyuan and Eugene Lim
%
% All the theorems are numbered according to <LINEAR ALGEBRA - Concepts and
% Techniques on Euclidean Space>, ISBN 978-981-3152-88-5, Second Edition (2016)
% -----------------------------------------------------------------------

\title{MA1101R Cheatsheet 19/20 Semester 1 Final}

\begin{document}

\begin{center}
{\large MA1101R Cheatsheet 19/20 Semester 1 Final}\\{by Howard Liu}
\end{center}

\footnotesize

\begin{multicols}{2}
\begin{justifying}

\setlength{\premulticols}{1pt}
\setlength{\postmulticols}{1pt}
\setlength{\multicolsep}{1pt}
\setlength{\columnsep}{2pt}

\section{Matrices}

\begin{namedthm*}{Theorem 1.2.7}
	If \textbf{augmented matrices} of two systems of linear equations are row equivalent, then the two systems have the same set of solutions. (\(\ast\) Even for two homogeneous linear systems, we still need to say that \(\begin{pmatrix}[c|c]\matr{A} & \matr{0}\end{pmatrix}\) is row equivalent to \(\begin{pmatrix}[c|c]\matr{B} & \matr{0}\end{pmatrix}\), not that \(\matr{A}\) is row equivalent to \(\matr{B}\).)
\end{namedthm*}

\begin{namedthm*}{Example 1.4.10}
	Suppose augmented matrix \(\matr{R}\) is in (R)REF:
	\begin{enumerate}
		\item LS has no solution \\
		    \(\Leftrightarrow\) Last column of \(\matr{R}\) is pivot.
		\item LS has one unique solution \\
		    \(\Leftrightarrow\) \textbf{Only} last column of \(\matr{R}\) is non-pivot.
		\item LS has infinite number of solution \\
		    \(\Leftrightarrow\) At least one column other than the last one is non-pivot \\
		    \(\Leftrightarrow\) Number of variables $>$ Number of non-zero rows in \(\matr{R}\) \\
		(\(\ast\) \# non-pivot columns in (R)REF \(- 1 =\) \# unique solutions)
	\end{enumerate}
\end{namedthm*}

\begin{namedthm*}{Theorem 6.1.8}
	\(\matr{A}\) is invertible when:
	\begin{enumerate}
		% T2.4.7
		\item \(\exists \matr{B}\) s.t. \(\matr{AB} = \matr{I} \lor \matr{BA} = \matr{I}\)
		\item Refer to \(\textbf{Theorem 2.4.7.2}\) below
		\item \(\rref(\matr{A}) = \matr{I}\)
		\item \(\matr{A}\) is a product of elementary matrices
		% T2.5.19
		\item \(\det(\matr{A}) \ne 0\)
		% T3.6.11
		\item Rows of \(\matr{A}\) is a basis of \(\mathbb{R}^n\)
		\item Columns of \(\matr{A}\) is a basis of \(\mathbb{R}^n\)
		% T6.1.8
		\item 0 is not an eigenvalue of \(\matr{A}\)
	\end{enumerate} 
\end{namedthm*}

\begin{namedthm*}{Remark 2.3.4 (Cancellation Laws for Matrices)}
	Let \(\matr{A}\) be an invertible \(m \times m\) matrix,
	\begin{enumerate}[label=(\alph*)]
		\item If \(\matr{B}_1\) and \(\matr{B}_2\) are \(m \times n\) matrices with \(\matr{AB_1} = \matr{AB_2}\), then \(\matr{B}_1 = \matr{B}_2\)
		\item If \(\matr{C}_1\) and \(\matr{C}_2\) are \(n \times m\) matrices with \(\matr{C_1A} = \matr{C_2A}\), then \(\matr{C}_1 = \matr{C}_2\)
	\end{enumerate}
\end{namedthm*}
\begin{namedthm*}{Theorem 2.4.7.2 (generalised)}
	Relationship between singularity of \(\matr{A}\) and the number of solutions of a linear system \(\matr{Ax} = \matr{b}:\)
	\begin{enumerate}
		\item \(\matr{A}\) is singular \(\Leftrightarrow \matr{Ax} = \matr{b}:\) has $\infty$ solutions (only case for homogeneous LS) or no solutions
		\item \(\matr{A}\) is invertible \(\Leftrightarrow \matr{Ax} = \matr{b}:\) has one unique solution (trivial solution for homogeneous LS)
	\end{enumerate} 
\end{namedthm*}

\begin{namedthm*}{Definition 2.5.2}
    Let \(\matr{A} = \left(a_{ij}\right)\) be an \(n \times n\) matrix. Let \(\matr{M}_{ij}\) be an \(\nobreak{(n - 1)\times (n - 1)}\) matrix obtained from \(\matr{A}\) by deleting the \(i\)th row and the \(j\)th column. Then the \textit{determinant} of \(\matr{A}\) is defined as
    \[
        \det(\matr{A}) =
            \begin{cases}
                a_{11} & \text{if \(n = 1\)} \\
                a_{11}A_{11} + \cdots + a_{1n}A_{1n} & \text{if \(n > 1\)}
            \end{cases}
    \]
    where
    \[
        A_{ij} = (-1)^{i + j} \det\left(\matr{M_{ij}}\right)
    \]
    The number \(A_{ij}\) is called the \((i, j)\)\textit{-cofactor} of \(\matr{A}\).
\end{namedthm*}

\begin{namedthm*}{Theorem 2.5.8}
    The determinant of a triangular matrix is equal to the product of its diagonal entries.
\end{namedthm*}

\begin{namedthm*}{Theorem 2.5.12 (added-on)}
	The determinant of a square matrix is 0 when:
	\begin{enumerate}
		\item it has two identical rows, or
		\item it has two identical columns
		\item any row/column of its (R)REF is zero
	\end{enumerate}
\end{namedthm*}

\begin{namedthm*}{Theorem 2.5.15}
    Let \(\matr{A}\) be a square matrix. \(k\) is a non-zero constant.
    \begin{enumerate}
    	\item \(\matr{A} \xrightarrow{k\vect{R}_i} \matr{B} \Rightarrow \det(\matr{B}) = k\det(\matr{A})\)
    	\item \(\matr{A} \xrightarrow{\vect{R}_i \leftrightarrow \vect{R}_j} \matr{B} \Rightarrow \det(\matr{B}) = -\det(\matr{A})\)
    	\item \(\matr{A} \xrightarrow{\vect{R}_i + k\vect{R}_j} \matr{B} \Rightarrow \det(\matr{B}) = \det(\matr{A})\)
        \item Let \(\matr{E}\) be an elementary matrix of the same size as \(\matr{A}\). Then \(\det(\matr{EA}) = \det(\matr{E})\det(\matr{A})\).
    \end{enumerate}
\end{namedthm*}

\begin{namedthm*}{Remark 2.5.18}
	Since \(\det(\matr{A}) = \det(A^T)\), theorem 2.5.15 holds if ``rows" are changed to ``columns".
\end{namedthm*}

\begin{namedthm*}{Theorem 2.5.22}
	Let \(\matr{A}\) and \(\matr{B}\) are two square matrices of order \(n\) and \(c\) is a scalar. Then
	\begin{enumerate}
		\item \(\det(c\matr{A}) = c^n\det(\matr{A})\)
		\item \(\det(\matr{AB}) = \det(\matr{A})\det(\matr{B})\)
		\item if \(\matr{A}\) is invertible, \(\det(\matr{A}^-1) = \frac{1}{\det(\matr{A})}\)
	\end{enumerate}
\end{namedthm*}

\begin{namedthm*}{Definition 2.5.24}
	Let \(\matr{A}\) be a square matrix of order \(n\). Then the \textit{(classical) adjoint} of \(\matr{A}\) is the \(n \times n\) matrix
	\[
	\adj(\matr{A}) = \left(A_{ij}\right)_{n \times n}^T
	\]
	where \(A_{ij}\) is the \((i, j)\)-cofactor of \(\matr{A}\).
\end{namedthm*}

\begin{namedthm*}{Theorem 2.5.27 (Cramer's Rule)}
    Suppose \(\matr{A}\vect{x} = \vect{b}\) is a linear system where \(\matr{A}\) is an \(n \times n\) matrix. Let \(\matr{A_i}\) be the matrix obtained from \(\matr{A}\) be replacing the \(i\)th column of \(\matr{A}\) by \(\vect{b}\). If \(\matr{A}\) is invertible, then the system has only one solution
    \[
        \vect{x} = \frac{1}{\det(\matr{A})}\begin{pmatrix}\det\left(\matr{A_1}\right) \\ \vdots \\ \det\left(\matr{A_n}\right) \end{pmatrix}
    \]
\end{namedthm*}

\begin{namedthm*}{Mixed Notes 1}
	\(\matr{A}^{-1}\) is able to be computed by:
	\begin{enumerate}
		\item Find \(\matr{B}\) s.t. \(\matr{AB} = \matr{I} \lor \matr{BA} = \matr{I}\)
		\item Find using \textbf{Theorem 2.5.25}: \(\matr{A}^{-1} = \frac{1}{\det(\matr{A})}\adj(\matr{A})\)
		\item Find using: \(\begin{pmatrix}[c|c] \matr{A} & \matr{I}\end{pmatrix} \xrightarrow{GJE} \begin{pmatrix}[c|c] \matr{I} & \matr{A}^{-1}\end{pmatrix}\)
	\end{enumerate} 
\end{namedthm*}

\begin{namedthm*}{Mixed Notes 2}
	\(\det(\matr{A})\) is able to be computed by:
	\begin{enumerate}
		\item Using \textbf{Theorem 2.5.2}
		\item Using cross multiplication (for \(2 \times 2\) and \(3 \times 3\) matrices only)
		\item Doing some ERO (e.g. GE, consider \textbf{Thoerem 2.5.15}) and making it triangular  then using \textbf{Theorem 2.5.8} or making it have properties in \textbf{Theorem 2.5.12}
		\item Using \textbf{Theorem 2.5.22}
	\end{enumerate} 
\end{namedthm*}

\begin{namedthm*}{Mixed Notes 3}
	Some random notes:
	\begin{enumerate}
		\item In \(\mathbb{R}^n\) where \(n \ge 2\), a set with 1 parameter is a line and that with 2 parameters is a space.
		\item \(\matr{M}^2 + \matr{M} = \matr{I} \Rightarrow \matr{M}(\matr{M} + \textcolor{red}{\matr{I}}) = \matr{I}\) (Don't put that \(\matr{I}\) to be scalar 1!)
		\item Two matrices have same RREF \(\Leftrightarrow\) They are row equivalent
		\item In exam, express a matrix in the form \(\matr{A} = (a_{ij})_{m \times n}\). \textbf{DO NOT} use dots form
		\item When using ERO \(\vect{R}_i = \frac{1}{k}\vect{R}_j\), discuss whether \(k\) is 0 when necessary
	\end{enumerate}
\end{namedthm*}

\begin{namedthm*}{Mixed Notes 4}
	Generally, for (square) matrices \(\matr{A}\) and \(\matr{B}\),
	\begin{enumerate}
		\item \(\matr{AB} \ne \matr{BA}\)
		\item \((\matr{AB})^2 \ne \matr{A}^2\matr{B}^2\)
		\item \(\matr{AB} = 0 \nRightarrow \matr{A} = 0 \lor \matr{B} = 0\)
		\item \(\matr{A}^2 = I \nRightarrow \matr{A} = \pm \matr{I}\) (For example: 2 EMs of 2nd type ERO)
	\end{enumerate}
\end{namedthm*}

\begin{namedthm*}{Mixed Notes 5}
	When expanding a row/column with cofactors of the other row/column, 0 will be yielded:
	\[
	    \sum_{m=1}^n a_{im}A_{jm} = \sum_{m=1}^n a_{mi}A_{mj} = 0, \text{ for some } i \ne j
	\]
\end{namedthm*}

\section{Euclidean Spaces}

\begin{namedthm*}{Discussion 3.2.5}
	Given \(S = \{\vect{v_1}, \vect{v_2}, \dots, \vect{v_m}\} \subseteq \mathbb{R}^n\}\), show \(\lspan(S) = \mathbb{R}^n\):
	
	\medskip
	\noindent
	Consider \(\vect{v_i} = \left(v_{i1}, \dots, v_{in}\right)\),
	\[
	\begin{pmatrix}
	\vect{v_{11}} & \dots & \vect{v_{m1}}\\
	\vdots & \ddots & \vdots\\
	\vect{v_{1n}} & \dots & \vect{v_{mn}}
	\end{pmatrix} \xrightarrow{GE} \matr{R}
	\]
	\(\lspan(S) = \mathbb{R}^n \Leftrightarrow \matr{R}\) has no zero rows
\end{namedthm*}

\begin{namedthm*}{Theorem 3.2.7}
	If \(|S| < n\), \(\lspan(S) \ne \mathbb{R}^n\).
\end{namedthm*}

\begin{namedthm*}{Theorem 3.2.10}
	Let \(S_1 = \{\vect{u_1}, \dots, \vect{u_k}\}\) and \(S_2 = \{\vect{v_1}, \dots, \vect{v_m}\}\) be subsets of \(\mathbb{R}^n\). Then, \(\lspan(S_1) \subseteq \lspan(S_2) \Leftrightarrow \forall i=1, 2, \dots, k\), \(u_i \in \lspan\{\vect{v_1}, \dots, \vect{v_m}\}\).
\end{namedthm*}

\begin{namedthm*}{Definition 3.3.2}
     Let \(V\) be a subset of \(\mathbb{R}^n\). Then \(V\) is called a \textit{subspace} of \(\mathbb{R}^n\) if \(V = \lspan(S)\) where \(S = \{\vect{u_1}, \dots, \vect{u_k}\}\) for some vectors \(\vect{u_1}, \dots, \vect{u_k} \in \mathbb{R}^n \).
     
     \medskip
     \noindent
     More precisely, \(V\) is called the \textit{subspace spanned} by \(S\) (or the \textit{subspace spanned} by \( \vect{u_1}, \dots, \vect{u_k} \)). We also say that \(S\) \textit{spans} (or \(\vect{u_1}, \dots, \vect{u_k}\) \textit{span}) the subspace \(V\).
     
     \medskip
     \noindent
     By contraposition, \(V = \lspan(S) \Rightarrow \vect{0} \in V \equiv \vect{0} \notin V \Rightarrow V \ne \lspan(S)\). (\(\ast\) i.e., If \(\vect{0}\) is not in \(V\), \(V\) is not a subspace of \(\mathbb{R}^n\))
\end{namedthm*}

\begin{namedthm*}{Theorem 3.3.6}
	If \(V = \{\matr{x} | \matr{Ax} = \matr{0}\}\), \(V\) is a subspace of \(\mathbb{R}^n\).
\end{namedthm*}

\begin{namedthm*}{Remark 3.3.8}
    Let \(V\) be a non-empty subset of \(\mathbb{R}^n\). Then \(V\) is a subspace of \(\mathbb{R}^n\) if and only if 
    \[
        \text{for all } \vect{u}, \vect{v} \in V \text{ and } c, d\in \mathbb{R},\enspace c\vect{u} + d\vect{v} \in V
    \]
    (\(\ast\) This checks whether V is \textbf{closed} under addition and scalar multiplication)
\end{namedthm*}

\begin{namedthm*}{Definition 3.4.2/4}
	Consider \(\vect{u_1}, \vect{u_2}, ..., \vect{u_k}\) which are column vectors, set \(S = {\vect{u_1}, \vect{u_2}, ..., \vect{u_k}}\) is \textbf{Linear Indepedent} iff. any of:
	\begin{enumerate}
		% T3.4.2
		\item \((\vect{u_1} \vect{u_2} ... \vect{u_k})\vect{x} = \matr{0}\) has only trivial solution.
		% T3.4.4
		\item No vectors in \(S\) can be written as a linear combination of other vectors in \(S\).
		% From revision note of tutor
		\item \(S\) is a subset of a \textbf{Linear Independent} set.
	\end{enumerate}
\end{namedthm*} 

\begin{namedthm*}{Definition 3.5.4/Theorem 3.6.7}
	A set \(S\) is a basis of a vector space if:
	\begin{enumerate}[label*=\arabic*.]
		\item \(S \subseteq V\)
		\item Any 2 of the 3 below:
		\begin{enumerate}[label*=\arabic*.]
			\item \(S\) is Linear Independent
			\item \(S\) spans \(V\)
			\item \(|S| = \dim(V)\)
		\end{enumerate}
	\end{enumerate}
\end{namedthm*}

\begin{namedthm*}{Definition 3.5.8}
	Let \(S = {\vect{u_1}, \vect{u_2}, ..., \vect{u_k}}\) be a basis for a vector space \(V\) and \(\vect{v}\) is a vector in \(V\). By T3.5.7, \(\vect{v}\) is expressed uniquely as a LC:
	\[
	    \vect{v} = c_1\vect{u_1} + c_2\vect{u_2} + \dots + c_k\vect{u_k}
	\]
	Then we shall have the \textbf{coordinate vector} of \(\vect{v}\) relative to the basis \(S\) : \((\vect{v})_S = (c_1, c_2, \dots, c_k) \in \mathbb{R}^k\) (assuming vectors in \(S\) are in fixed order).
\end{namedthm*}

\begin{namedthm*}{Remark 3.5.10/Theorem 3.5.11}
	Let \(S\) be a basis for a vector space \(V\),
	\begin{enumerate}
		\item \(\forall \vect{u}, \vect{v} \in V, \vect{u} = \vect{v} \Leftrightarrow (\vect{u})_S = (\vect{v})_S\)
		\item Coordinate vectors are closed under scalar multiplication and addition
		\item Let \(\vect{v_1}, \vect{v_2}, \dots, \vect{v_r} \in V\), they are LI iff. \((\vect{v_1})_S, (\vect{v_2})_S, \dots, (\vect{v_k})_S\) are LI
		\item \(\lspan {\vect{v_1}, \vect{v_2}, \dots, \vect{v_r}} = V \Leftrightarrow \lspan {(\vect{v_1})_S, (\vect{v_2})_S, \dots, (\vect{v_k})_S} = \mathbb{R}^{|S|}\)
	\end{enumerate}
\end{namedthm*}

\begin{namedthm*}{Theorem 3.6.9}
	Let \(U\) be a subspace of \(V\), then \(\dim(U) \le \dim(V)\). Furthermore, if \(U \ne V\), then \(\dim(U) < \dim(V)\).
\end{namedthm*}

\begin{namedthm*}{Definition 3.7.3}
	Let \(S = {\vect{u_1}, \vect{u_2}, ..., \vect{u_k}}\) and \(T\) be two bases for a vector space. The square matrix \(\matr{P} =\begin{pmatrix}{[\vect{u_1}]}_T & {[\vect{u_2}]}_T & \dots & {[\vect{u_k}]}_T\end{pmatrix}\) is called the \textbf{transition matrix} from \(S\) to \(T\).
\end{namedthm*}

\begin{namedthm*}{Mixed Theorem 6}
	Consider \(S\) and \(T\) are two bases for vector space \(V\) and \(\matr{P}\) is the transition matrix from \(S\) to \(T\). If \(\matr{A}\) and \(\matr{B}\) are matrices with elements of \(S\) and \(T\) respectively as columns, we have \(\matr{BP} = \matr{A}\).
\end{namedthm*}

\begin{namedthm*}{Mixed Theorem 7}
	ERO preserves row space \textbf{(T4.1.7)}, and we have:
	\begin{itemize}
		\item \textbf{(R4.1.9)} \(\matr{R}\) is RREF of \(\matr{A}\). Non-empty rows in \(\matr{R}\) forms the basis of row space of \(\matr{A}\).
		\item \textbf{(T4.2.1)} Row space and column space of a matrix have the same dimension.
	\end{itemize}
\end{namedthm*}

\begin{namedthm*}{Remark 4.2.5}
	Regarding rank(\(\matr{A}\)):
	\begin{enumerate}
		\item For \(m * n\) matrix \(\matr{A}\), \(\rank(\matr{A}) \le \min{m, n}\). If \(\rank(\matr{A}) = \min{m, n}\), \(\matr{A}\) is said to have \textbf{full rank}.
		\item A square matrix \(\matr{A}\) have full rank iff. it is invertible.
		\item \(\rank(\matr{A}) = \rank(\matr{A^T})\).
	\end{enumerate}
\end{namedthm*}

\begin{namedthm*}{Theorem 4.3.6}
	Suppose linear system \(\matr{A}\vect{x} = \vect{b}\) has solution \(\vect{v}\), then the solution set of this system is given by:
	\[
	    M = \{ \vect{u} + \vect{v} | \vect{u} \in \text{nullspace\((\matr{A})\)}\}
	\]
\end{namedthm*}

\section{Orthogonality}

\begin{namedthm*}{Definition 5.1.2.3/4}
	For two vectors \(\vect{u}\) and \(\vect{v}\):
	
	\(d(\vect{u}, \vect{v}) = \norm{\vect{u} - \vect{v}}\).
	
	Angle between \(\vect{u}\) and \(\vect{v}\) is:
	\[
	    \cos^{-1}(\frac{\vect{u}\cdot\vect{v}}{\norm{\vect{u}}\norm{\vect{v}}})
	\]
\end{namedthm*}

\begin{namedthm*}{Theorem 5.2.4}
	If \(S\) is an orthogonal set of non-zero vectors in a vector space, \(S\) is \textbf{LI}.
\end{namedthm*}

\begin{namedthm*}{Theorem 5.2.8}
	Consider \(S = \{\vect{u_1}, \vect{u_2}, \dots, \vect{u_k}\}\) is a basis for a vector space \(V\), then for any vector \(\vect{w}\) in \(V\):
	\begin{enumerate}
		\item If \(S\) is orthogonal, we have
		\[
		    (\vect{w})_S = (\frac{\vect{w}\cdot\vect{u_1}}{\vect{u_1}\cdot\vect{u_1}}\vect{u_1}, \frac{\vect{w}\cdot\vect{u_2}}{\vect{u_2}\cdot\vect{u_2}}\vect{u_2}, \dots, \frac{\vect{w}\cdot\vect{u_k}}{\vect{u_k}\cdot\vect{u_k}}\vect{u_k})
		\]
		\item If \(S\) is orthonomal, we have
		\[
		    (\vect{w})_S = (\vect{w}\cdot\vect{u_1}, \vect{w}\cdot\vect{u_2}, \dots, \vect{w}\cdot\vect{u_k})
		\]
	\end{enumerate}
    \textbf{T5.2.15}: \((\vect{w})_S\) is the projection of \(\vect{w}\) onto \(V\) if \(\vect{w} \in \mathbb{R}^n \land V\) is a subspace of \(\mathbb{R}^n\) (condition of \(\vect{w}\) changed but same formula applies).
\end{namedthm*}

\begin{namedthm*}{Theorem 5.2.19 (Gram-Schmidt Process)}
	Let {\(\vect{u_1}, \vect{u_2}, \dots, \vect{u_k}\)} be a basis for a vector space \(V\). Let
	\begin{gather*}
	    \vect{v_1} = \vect{u_1},\\
	    \vect{v_2} = \vect{u_2} - \frac{\vect{u_2}\cdot\vect{v_1}}{\vect{v_1}\cdot\vect{v_1}}\vect{v_1},\\
	    \vect{u_3} = \vect{u_3} - \frac{\vect{u_3}\cdot\vect{v_1}}{\vect{v_1}\cdot\vect{v_1}}\vect{v_1} - \frac{\vect{u_3}\cdot\vect{v_2}}{\vect{v_2}\cdot\vect{v_2}}\vect{v_2},\\
	    \vdots
	\end{gather*}
	Then {\(\vect{v_1}, \vect{v_2}, \dots, \vect{v_k}\)} is an orthogonal basis for \(V\). Normalize all vectors in it then we have a orthonormal basis for \(V\).
\end{namedthm*}

\begin{namedthm*}{Definition 5.3.6}
	Let \(\matr{A}\vect{x} = \vect{b}\) be a linear system where \(\matr{A}\) is an \(m * n\) matrix. A vector \(\vect{u} \in \mathbb{R}^n\) is called a \textbf{least squares solution} to the linear system if \(\forall \vect{u} \in \mathbb{R}^n, \norm{\vect{b} - \matr{A}\vect{u}} \le \norm{\vect{b} - \matr{A}\vect{v}}\).
\end{namedthm*}

\begin{namedthm*}{Theorem 5.3.8}
	Continuing \textbf{D5.3.6}, let \(\vect{p}\) be the projection of \(\vect{b}\) onto the column space of \(\matr{A}\). \(\vect{u}\) is the least squares solution iff. \(\matr{A}\vect{u} = \vect{p}\).
\end{namedthm*}

\begin{namedthm*}{Theorem 5.3.10}
	Continuing \textbf{D5.3.6}, \(\vect{u}\) is the least squares solution iff. \(\vect{u}\) is a solution to \(\matr{A}^T\matr{A}\vect{x} = \matr{A}^T\vect{b}\).
\end{namedthm*}

\begin{namedthm*}{D5.4.3/R5.4.4/T5.4.6}
	\(\matr{A}\) is a square matrix of order \(n\). The following are equivalent:
	\begin{enumerate}
		\item \(\matr{A}\) is orthogonal
		\item \(\matr{A}^{-1} = \matr{A}^T\)
		\item \(\matr{A}\matr{A}^T = \matr{A}^T\matr{A} = \matr{I}\)
		\item The rows of \(\matr{A}\) form an \textbf{orthonormal} basis for \(\mathbb{R}^n\)
		\item The columns of \(\matr{A}\) form an \textbf{orthonormal} basis for \(\mathbb{R}^n\)
	\end{enumerate}
\end{namedthm*}

\begin{namedthm*}{Theorem 5.4.7}
	Let \(S\) and \(T\) be two \textbf{orthonormal} bases for a vector space and let \(\matr{P}\) be the transition matrix from \(S\) to \(T\). Then \(\matr{P}\) is orthogonal and \(\matr{P}^T\) is the transition matrix from \(T\) to \(S\).
\end{namedthm*}

\section{Diagonalization}

\begin{namedthm*}{Definition 6.1.3}
	\(\matr{A}\) is a square matrix of order \(n\). \(\vect{u} \in \mathbb{R}^n\) is an non-zero column vector that satisfies:
	\[
	    \matr{A}\vect{u} = \lambda \vect{u}
	\]
	for some scalar \(\lambda\). \(\lambda\) is called an \textbf{eigenvalue} of \(\matr{A}\). \(\vect{u}\) is said to be an \textbf{eigenvector} of \(\matr{A}\) \textbf{associated} with the eigenvalue \(\lambda\).
\end{namedthm*}

\begin{namedthm*}{Theorem 6.1.9}
	If \(\matr{A}\) is triangular, the eigenvalues of \(\matr{A}\) are the diagonal entries of \(\matr{A}\).
\end{namedthm*}

\begin{namedthm*}{Remark 6.2.5}
	Suppose the characteristic polynomial of the matrix \(\matr{A}\) can be factorized as
	\[
	    \det(\lambda\matr{I} - \matr{A}) = (\lambda - \lambda_1)^{r_1}(\lambda - \lambda_2)^{r_2}\dots(\lambda - \lambda_k)^{r_k}
	\]
	where \(\lambda_1, \lambda_2, \dots, \lambda_k\) are distinct eigenvalues of \(\matr{A}\). Then for each eigenvalue \(\lambda_i\), 
	\[
	    \dim(E_{\lambda_i}) \le r_i
	\]
	Furthermore, \(\matr{A}\) is diagonalizable iff. \(\forall 1 \le i \le k, \dim(E_{\lambda_i}) = r_i\).
\end{namedthm*}

\begin{namedthm*}{Definition 6.3.2/T*.4}
	A square matrix \(\matr{A}\) is said to be orthogonally diagonalizable iff. there exists an orthogonal matrix \(\matr{P}\) such that \(\matr{P}^T\matr{A}\matr{P}\) is diagonal.
	
	A square matrix is orthogonally diagonalizable iff. it is \textbf{symmetric}.
\end{namedthm*}

\begin{namedthm*}{Algorithm 6.3.5}
	Similar to the process for the normal matrix, orthogonal matrix \(\matr{P}\) can be found by using vectors of \(T\) as \textbf{its columns} where \(T = T_{\lambda_1} \cup T_{\lambda_2} \cup \dots \cup T_{\lambda_k}\) and \(T_{\lambda_i}\) is transformed from \(S_{\lambda_1}\) using Gram-Schmidt Process.
\end{namedthm*}

\section{Linear Transformation}

\begin{namedthm*}{Theorem 7.1.4}
	Let \(T\) be a linear transformation, we have:
	\begin{enumerate}
		\item \(T(\vect{0}) = \vect{0}\)
		\item \(T\) is closed under scalar multiplication and addition
	\end{enumerate}
\end{namedthm*}

\begin{namedthm*}{Discussion 7.1.8}
	Let \(T: \mathbb{R}^n \rightarrow \mathbb{R}^m\) be a linear transformation with the standard matrix \(\matr{A}\). Let \(\{\vect{e_1}, \vect{e_2}, \dots, \vect{e_n}\}\) be the standard basis for \(\mathbb{R}^n\). We then have:
    \[
        \matr{A} = 
        \begin{pmatrix}
        T(\vect{e_1}) & T(\vect{e_2}) & \dots & T(\vect{e_n})
        \end{pmatrix}
    \]
\end{namedthm*}

\begin{namedthm*}{Theorem 7.2.4}
	Continuing \textbf{D7.1.8}. We have:
	\[
	    R(T) = \lspan\{T(\vect{e_1}), T(\vect{e_2}), \dots, T(\vect{e_n})\} = \text{the column space of }\vect{A}
	\]
	which is a subspace of \(\mathbb{R}^m\)
\end{namedthm*}

\begin{namedthm*}{D7.2.5/T7.2.9/D7.2.10/T7.2.12}
	Continuing \textbf{T7.2.4}. We have:
	\begin{itemize}
		\item \(\rank(T) = \dim(R(T)) = \rank(\matr{A})\)
		\item \(\nullity(T) = \nullity(\matr{A})\)
		\item \(\rank(T) + \nullity(T) = n\)
		\item \(\ker(T) = \text{the nullspace of }\matr{A}\)
	\end{itemize}
\end{namedthm*}
\end{justifying}
\end{multicols}

\end{document}
